\section{Methodology}
The project can be divided into three sections: research, development and testing, with development being further subdivided into the initial backbones of the software, and then repeated strategy implementation.


\subsection{Research}
The initial research will be into creating a thorough understanding of the technologies and development tools. The frontend will utilise Next.js, a react framework with faster page loading times making the user experience better \cite{NextJSSpeed}; and the backend will be written in JavaScript with express, and use a Mongo database to store user data \cite{MongoDB}. In order to use the Plaid APIs with the correct flow for authentication, the frontend requests must go through the backend as a proxy before reaching the plaid servers to process the request and return the data. 

Beyond the initial research, during the repeated strategy implementation, academic papers, books, reports and other sources will be read in order to discover the money management strategies. Furthering this, within each strategy there may be conflicts and sub-strategies which must be weighed up and analysed to decide which is the best to add to the software. The intention of the research is so that the software has methods, with evidence to support these, that actually benefit a user in taking control of their finances; not just a clone of the existing tools. A trivial example is that people often have several bank accounts, each for different purposes. In order to properly view all the transactions and determine where money could be saved, they ought to be collated into one central view \cite{HavingMultipleAccounts}. Further research will be done into finding evidence and reasoning to support that this is true, but if not and a better solution is found, then that idea will be implemented into the tool.

\subsection{Development}
The initial backbones of the website will be developed using the waterfall strategy, as the plan-based methodology lends itself nicely to getting the basic software ready for further development \cite{WaterfallVsAgile}. Full design, implementation, and testing will be done on this skeleton to confirm it works as expected. This foundation involves having full user authentication, with signing up and logging in, a basic dashboard where users can link their banks, and a page to view all their most recent transactions from any of their accounts. The benefit of having a clear end goal makes the waterfall strategy perfect as detailed planning can be completed beforehand, followed by methodical progression through the stages. Moreover, it eases time management as set lengths of time can be accurately estimated for each part, helping ensure the project's objectives are reached.

During the repeated strategy development phase, an agile approach will be adopted. This synergizes with the aim to do large amounts of research on effective strategies for individuals to increase their financial capability; followed by the implementation of that strategy into the website. The features that will be added are currently unknown so are unable to be planned. Additionally, the fact that at this stage the process will be quite repetitive, works well with the cyclic nature of agile's of planning, designing, developing and testing sprints \cite{WaterfallVsAgile}. At each cycle, the software should be ready to be distributed so a working version is always ready.

Throughout development, git will be used to as version control to manage changes and features effectively \cite{Git}. If a mistake is ever made, reverting to a working version can be done quickly. The Git flow practise will also be followed, using main, release, development and feature branches, along with frequent merges to keep the repository clean and professional \cite{GitFlow}. Finally, GitHub will be used to store a remote version of the repository and also aid documentation \cite{GitHub}.

\subsection{Testing}
Similar to the development phase, the testing will be split up into two sections. Initially when working on the website's foundation, frequent unit testing will be done on the frontend and backend. In some instances, the tests will be written before the code is, as this follows the test-driven development style to help break down the problem into smaller manageable chunks, as well as improves code quality in general \cite{TDD}. Some of the tests will cover the integration of the frontend and backend, for example when a user signs up, the backend will add the user to the database and the frontend update the state to reflect this.

In the second phase of development, the testing will be more user-based and occur more frequently due to the testing sections of each sprint in agile development. Following the implementation of a strategy, a tester will be asked to use that strategy and see it works as expected. A list of steps will not be given in an attempt to also test the how intuitive and easy to navigate the UI is. The user's feedback will be acted upon and the feature will be improved if necessary.

After the major development has finished, the software will be tested by a larger group of users with various levels of experience. This will be more use-case testing, where the users are allowed to experiment with the software as they would normally, and report any issues they find \cite{UseCaseTesting}. The results of this testing will be analysed and any issues will be fixed before the software is released.