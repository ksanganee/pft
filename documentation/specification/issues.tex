\section{Legal, Social, Ethical and Professional Issues}
The software will have an authentication system enabling users to sign up and log in, making the user experience smooth. Consequently, the current guidance and best practices for data security will have to be followed. This includes enforcing a long and complex password for users to avoid unauthorised access to accounts \cite{PasswordRequirements}, but also storing user's passwords and data in a fashion which complies with the GDPR \cite{GDPR}.

As a result of using the Plaid API, no extremely sensitive banking information will be stored locally; All the financial data is handled by the Plaid service itself. When the data is displayed to the user, they can be sure that it is secured and accurate, so long as the correct flow that is described in the Plaid documentation is followed during development \cite{PlaidGettingStarted}. Furthermore, when accessing certain products provided by Plaid, steps will be taken to avoid getting unnecessary data, for example, when accessing an account's transactions, the account number can be displayed as several asterisks followed by only the last few digits in plaintext, in an attempt to avoid vulnerabilities.

The biggest issue the software must avoid is ensuring that the strategies and suggestions are not considered financial advice. To mitigate this, research will be done into knowing the boundary between advice and guidance. The functionality on the website which helps individuals become more financially capable must be opt in, as to avoid pushing suggestions onto users because it may overwhelm them and is against the laws set out by the Financial Conduct Authority \cite{AdviceVsGuidance}.