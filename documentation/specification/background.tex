\section{Background}
The existing software can be categorised into three sections, all of which have their own problems. 

Firstly, there are the mobile banking apps, such as Revolut and Monzo, which track customers' spending within those specific accounts \cite{RevolutBudgetting} \cite{MonzoBudgetting}. They often struggle to, or don't even at all, integrate across other banks to have a singular centralised view and force users to create bank accounts with them to use. In addition, users must download a mobile app, which they are less likely to trust with their personal data, rather than just accessing a website \cite{AppVsWebsite}.

Then there are the websites which allow users to sync data with their banks. After signing up, users allow the website to access their bank accounts, which then very slowly imports all their data. Navigating on these is unintuitive and eventually, after a user would find a service which doesn't cost a fortune, the obstructive adverts will pop up and block them from even using it \cite{PersonalFinanceAppsComparison}. In addition, some sites are known to not have enough features to make them worth using, and others have too many useless features which don't actually help, and instead overwhelm the user.

The third category is the websites or apps which do not let users integrate their accounts. These expect users to manually add their banking information and then they perform analysis to provide suggestions on how to save money. They are barely better than an Excel spreadsheet and are by far the worst of the three categories.

Ideally, the software will solve most of the problems with the existing solutions, incorporate the best features they have into one, and, have additional functionalities which are deemed to be helpful in improving money management found from the research.

In trying to avoid the category of software which doesn't let individuals sync across their banking information, the project will utilise the data from endpoints as a result of the open banking movement. In particular, it will use the Plaid API \cite{Plaid}, enabling the solution to obtain information from all bank accounts that a user gives permission to access, and compiles it into one centralised location for easy analysis and better results.                                                                           