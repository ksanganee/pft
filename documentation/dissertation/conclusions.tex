\chapter{Conclusions}
\label{ch:conclusions}

\section{Summary}
Overall, the project was a success as the initial requirements were met and success in completing the title of this dissertation. A performant web application was made with minimalism and elegance in mind. The website features live transactional information in the form of strategies which aim to help increase financial capability. An overview of all areas of a user's finances will help inform their better decisions. The ability to toggle accounts on and off is a new feature not found in current personal finance tools. The proper Plaid authentication flow was followed with a clear separation between the frontend and backend. An accurate, by some measure, machine learning model was built to make expenditure predictions for all users. Finally, this was all incorporated, with the database, into a single deployable web application bundle.

\section{Future Work}
Follow-up work that could continue from the end of this project includes building further strategies. More strategies would mean more information is available for the user. There would still be an emphasis on avoiding complexity in an attempt not to confuse the user, so these would have to be carefully considered, but there is potential for more.

In addition to this, Plaid also offer a money transfer service between bank accounts. A potential feature that could be included in the application would be the ability to make transactions between linked accounts. For example, a user could discover that they have surplus money in their spending account, and then using the tool, they could transfer this money to their savings account.

Finally, after developing a web application, the next step is maintaining it. The application will need to be kept up to date with security measures, as it handles sensitive data, and fix any bugs that may arise. In addition, new and better libraries are often released, so there is always the potential to simplify and improve the technologies used in this application.