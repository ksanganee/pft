\chapter{Introduction}
\label{ch:introduction}
The term `personal finance tool' encompasses various software, such as budgeting tools, investment management software and credit score calculators. This project aims to investigate strategies that help build a user's financial capability and confidence, followed by implementing a web application to support these strategies. In particular, the research will focus on strategies to help individuals to manage their, often several, bank accounts in one place.

\section{Financial Capability}
Being financially capable is defined by the Financial Educators Council as having the "skills and knowledge of financial matters to confidently take effective action that best fulfils an individual's personal, family and global community goals" \cite{FinancialEducatorsCouncil}. Consequently, the application aims to give users the tools to view their financial circumstances as part of the knowledge, and with it comes the confidence to make informed, beneficial decisions.

Financial capability and financial literacy will be used interchangeably for the remainder of this document.

\section{Motivation}
Part of the motivation for this project comes from the recent open banking technology movement \cite{OpenBanking}. This movement is where thousands of major banks have opened a set of endpoints allowing third-party applications to access their customers' financial data with explicit consent securely. It allows developers to build personalised financial applications and services tailored to their users, which is most appropriate for this project. Further details and other motivations are discussed in the background chapter below (\ref{sec:further-motivation}).
