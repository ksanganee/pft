\chapter{Introduction}
\label{ch:introduction}

The term personal finance tool encompasses a wide range of software, including budgeting tools, investment management software and credit score calculators. This project aims to investigate strategies which help build a user's financial capability and confidence; followed by the implementation of a web application to support these strategies. In particular, the research focused on strategies to help individuals to manage their, often several, bank accounts and expenses in one place.

Being financially capable is defined by the Financial Educators Council as having the "skills and knowledge of financial matters to confidently take effective action that best fulfills an individual's personal, family and global community goals" \cite{FinancialEducatorsCouncil}. The website therefore aims to give user's the tools to view their financial circumstances, as part of the knowledge, and with it comes the confidence to make informed beneficial decisions. For the rest of this document, financial capability and financial literacy will be used interchangeably.

Part of the motivation for this kind of project comes from the recent open banking technology movement \cite{OpenBanking}. This is where thousands of major banks have opened a set of endpoints to allow third party applications to securely access their customers' financial data with authorisation. It allows developers to build personalised financial applications and services that are tailored to the needs of the user, based on this data; and is most appropriate for this project. Further detail and other motivations are discussed in the background chapter below (\ref{ch:background}).


% \section{Related work}

% Discuss related work.

% \section{Objectives}

% One sentence summary of your project. Followed by a short list of concrete objectives:

% \begin{itemize}
%     \item Objective 1
%     \item Objective 2
%     \item Objective 3
% \end{itemize}