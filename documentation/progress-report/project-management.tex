\section{Project management}

As alluded to in the previous section, the progress made so far is in accordance with the project timeline set out in the specification, therefore the timetable remains unchanged. Ideally, the full proof of concept application will be finished by the end of term 1, however due to the large amount of other coursework deadlines, this can be extended to the beginning of term 2 and not affect the rest of the timeline.

Following the proof of concept, the repeated strategy implementation phase begins. This is where in cycles of three weeks. The first week is dedicated to research and planning in which the best strategy for that feature will be chosen e.g. if the focus is investments, then research will be done into how investments affect a user's financial capability and what would aid them. The second week is for development where this chosen approach for that strategy is implemented into the application. This will be done as a separate git branch in accordance with the git flow feature branch practice \cite{GitFlow}. The third week is testing and bug fixing before being merged into the main application and the cycle begins again. This is appropriate as the requirements at this stage are less clear and quite open so need to be flexible. Following this, the application will be tested more thoroughly and focus will shift to writing the dissertation.