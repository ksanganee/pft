\section{Background}
\label{sec:background}

The idea of an individual being financially capable is defined by the Financial Capability Strategy for the UK as them having the ``ability to manage money well'' and the ``attitude ...\ [and] confidence to put (their) money skills into practise'' \cite{FinancialCapabilityUK}. After researching what tools will be best to help users achieve this, they will be implemented into the application. This will not only give users confidence, but also the information they need to improve their money management skills and make the right decisions.


\subsection{Limitations of Current Applications}
Motivation for such software comes from the limitations in the current applications available to users. The project specification (Appendix: \ref{appendix:project-specification}) goes into more detail but in summary, there are three different types of software. Firstly, there are mobile banking apps which track customers' spending on that account, however, these these do not integrate with any accounts from different companies. Then, there are websites which perform the analysis on various accounts but require the user to manually import their financial data. Finally, there are websites which automatically sync the data, but they lack many saving features, are difficult to use, and often require a paid subscription \cite{PersonalFinanceAppsComparison}.

\subsection{Software}
This project is an attempt to solve these problems. The resulting software will be a web application which can import users' financial data securely and automatically. This is done with the Plaid API \cite{Plaid}, a set of endpoints to access a user's bank account data as a result of the recent open banking movement \cite{OpenBanking}. A secure authentication system will be implemented to ensure that only the correct user can access their data. After signing in, the user will be able to link any of their bank accounts for analysis to be performed. Their dashboard will include all their recent transactions, and they will be able to switch to a category view to understand their spending habits. Further features and analysis tools will be implemented based on the research into which strategies are most effective. Examples include a budgeting section to organise a budget and encourage saving; as well as an investments tab where users can see their current investments and the performance of their portfolio.

\subsection{UK Cost of Living Crisis}
The timing of this project is also extremely appropriate as the UK is currently facing a cost of living crisis and is only made worse by the war in Ukraine \cite{CostOfLivingCrisisArticle}. The inflation rate of the household utility sector is at 26.6\% according to the UK Office for National Statistics \cite{CPISectorStats}. Users could make use of the software as they can group their transactions by category, enabling them to better compare their bills and see where they can save. Furthermore, by using the budgeting feature, they can set strict limits on what is spent and what is saved to be in a more comfortable position in the future.

